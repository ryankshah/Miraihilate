\chapter*{Abstract}

\begin{flushleft}

A botnet is a network of infected devices, that are usually remotely controlled, used to perform large-scale attacks such as a distributed denial-of-service (DDoS) attack and conduct other illegal activities.

\vspace{0.5cm}\par

Mirai is a self-propagating, IoT ("Internet of Things") botnet virus that infects out-of-date and vulnerable IoT devices running the Linux operating system. It does this by identifying a vulnerable device by rooting into them using a list of common default usernames and passwords. These infected devices are now effectively "bots" and are under control by the command and control program that spreads the virus. They will continue to run as normal until asked to perform a specific task. It is an effective botnet as these IoT devices are vulnerable and easily infected and for example, since there are so many of them, it is possible to generate enormous amounts of throughput.

\vspace{0.5cm}\par

In this dissertation, I will investigate why some state that the Mirai botnet is more effective than other major botnets in the past and why it is effectively unstoppable. I will be creating a program which will allow system administrators to scan their networks for potential vulnerable devices, and attempt to patch the device. Whilst comparing the program's ability to prevent Mirai through a known vulnerability with other programs that attempt to stop Mirai, I will be evaluating how effective it is for helping system administrators patch this vulnerability.

\end{flushleft}
