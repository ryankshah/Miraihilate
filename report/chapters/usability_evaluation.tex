\chapter{Usability Evaluation}

Along with the feedback from the mockups before implementation, I performed a usability test on an initial working prototype of Miraihilate. The aim of this study was to gain final usability feedback before finalising the program prior to testing the final working version.

\section{Recruitment}

In order to conduct a reliable user evaluation I aimed to test at least 20 people, due to the study having both qualitative and quantitative data being collected. Because Miraihilate will initially consist of two user roles (operators and administrator), I will design questions to test the usability of both the roles and select participants with at least a moderate technical background

\vspace{0.5cm}

For selecting participants, I used global design company IDEO's method of recruiting "\textit{Extreme}" and "\textit{Mainstream}" participants. Extreme partipants allow me to include participants that cover skills included in my target user group, whilst having mainstream participants allows me to also include participants who have minimal computing experience. If the mainstream participants do not have much difficulty with the mockups, it suggests that the majority of participants will be able to perform well on the study as well. To identify the extreme and mainstream candidates, before conducting the study they answered a brief question sheet consisting of questions to analyse their technical capability, so that their results can be categorised into what I would consider extreme and mainstream users. The brief question sheet used for categorising participants can be found in Appendix C.2.

\section{Conducting the Study}

Before the study commenced, participants were asked to sign a consent form (shown in Appendix C.1) to the study, explaining what the purpose of the study was and what would happen. It also identified that they can opt-out of the study at any time and their results will be withdrawn.
As described in the recruitment section, participants were first given a brief question sheet to analyse their technical capability. This had no effect on the study itself, but was used in the analysis of results to categorise the participants into extreme and mainstream users.

\vspace{2cm}

After this, participants were then given the program to use alongside four scenario questionnaires (found in appendices C.3-6). The scenarios given to participants are:
\begin{enumerate}
	\item{Setting up Miraihilate}
	\item{Performing a quick scan and viewing recent scan history}
	\item{Changing their password from the auto-generated one}
	\item{Performing an advanced scan with extra commands from a given list}
\end{enumerate}

\vspace{0.5cm}

The first scenario involves a user running the setup utility from the commandline from a set of instructions, which downloads the program and sets up their own admin account to use for the next three scenarios. The second involves using another set of instructions to perform a quick scan on a local network setup prior to the study, and then viewing the latest scan result by opening up the scan history window. Thirdly, participants will be asked to edit their profile by changing their password. If they wish to change anything else they have an option to do so. Finally participants, having already experienced the quick scan, will perform an advanced scan. They will be given a list of commands that they must choose at least one from, to use in the extra commands option in the scan window. Upon completing the scenario questionnaires, participants were asked verbally how they felt the study went.

\section{Results Analysis}
