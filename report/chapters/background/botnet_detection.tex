\raggedright

Detecting botnets has proven to be a difficult task, as bots are designed to infect and operate without user consent or knowledge. However, they are not completely invisible\textsuperscript{\cite{lee2007botnet}}.

\vspace{0.5cm}

One indicator that a device may be part of a botnet involves extremely high CPU usage or slow computing speeds. Problems with Internet access can also be another indicator alongside this. Preventing this issue can include keeping all software up-to-date with their respective security patches, as well as monitoring network activity so that irregular behaviour can be identified\textsuperscript{\cite{gu2008botsniffer}}.

\vspace{0.5cm}

Outgoing spikes in traffic from specific ports can also be monitored to detect possible botnet activity. Ports such as \textit{Port 6667} (IRC), \textit{Port 25} (Email Spam) and \textit{Port 1080} (Proxy Servers) are usually the most used.

As well as this high outgoing SMTP traffic, found as a result of transmitting spam, is another possible indicator of infections.

\vspace{0.5cm}

With these detection procedures in mind however, symptoms that are indicative of bot infections can also be signs of regular malware infections or network problems. Therefore, they should not be taken as a sure sign that a device has been compromised by bot malware.

There is no sure fire way of preventing being infected, however regulating network activity, keeping programs up-to-date and being vigilant whilst using the internet are preventative methods against being infected by bot malware.

Anti-botnet tools provide botnet detection procedures by finding and blocking known bot viruses before infection can occur. Network-based and Host-based \textit{Intrusion Detection Systems} (NIDS/HIDS) and network sniffers, as well as anti-bot programs can provide more sophisticated measures for detection, prevention and removal of bot malware, and provide security policies on outgoing communication from the IoT devices\textsuperscript{\cite{braun2017system}}.

\vspace{0.5cm}

For organisations, on a larger scale, removing botnet malware often requires disabling or taking down the command and control server operating the botnet. An example of this being done in the past was Microsoft's campaign against the Zeus botnet\textsuperscript{\cite{zeusbotnet}}.