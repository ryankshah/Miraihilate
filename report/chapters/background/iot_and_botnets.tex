The increasing popularity of IoT devices, especially in the home, have made them a popular target and powerful amplification platform for cyber-attacks\textsuperscript{\cite{kolias2017ddos}}.

\vspace{0.5cm}

In September 2016, a network security blog owned by Brian Krebs was hit with 620Gbps of traffic. At around the same time an even bigger attack was performed peaking at 1.1Tbps, making it one of the biggest attacks ever performed. Both of these attacks were performed by an IoT botnet named Mirai.

\vspace{0.5cm}

\subsection{Vulnerability}

Amongst these major attacks, there has been a frequent recurrence of attacks involving IoT devices, clearly demonstrating they are extremely vulnerable in terms of security. The large quantity of these devices around the world, as well as being extremely vulnerable, has attracted a new era of botnet malware.

It comes down to three primary factors, which determine the vulnerability of IoT devices:
\begin{itemize}
	\item{An embedded or simplified OS, such as Linux, which is comparatively easier to compromise}
	\item{Lack of capacity and processing power for standard security capabilities}
	\item{Aspects of hardware and software are reused to save engineering time, resulting in default passwords}
\end{itemize}